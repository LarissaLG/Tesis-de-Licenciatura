\documentclass[11pt,letterpaper]{article}
\usepackage[utf8]{inputenc}
\usepackage[spanish]{babel} 
\decimalpoint

%----- Configuración del estilo del documento------%
\usepackage{epsfig,graphicx}
\usepackage[left=2cm,right=2cm,top=1.8cm,bottom=2.3cm]{geometry}
\usepackage{fancyhdr}
\pagestyle{fancy}
\fancyhf{}
\fancyfoot{}
\graphicspath{{Figuras/}}
\usepackage{hyperref}
\hypersetup{colorlinks=true,linkcolor=blue,citecolor=blue,filecolor=blue,urlcolor=magenta,}

\fancyfoot[RO,LE]{\thepage} % Custom footer text
\fancyheadoffset[RO,LE]{0.01\textwidth}

%------ Paquetes matemáticos básicos --------%
\usepackage{amsmath}
\usepackage{amssymb}
\usepackage{amsthm}

%------ Texto aleatorio ----- %

\usepackage{lipsum}


\usepackage[bibstyle = trad-abbrv,
citestyle = numeric-comp,
backend = bibtex, 
sorting = none, % Sort as they appear
backref=false]{biblatex} %style = trad-abbrv 
\AtEveryBibitem{\clearfield{urldate}}
\AtEveryBibitem{\clearfield{url}}
\AtEveryBibitem{\clearfield{isbn}}
\AtEveryBibitem{\clearfield{month}}
\AtEveryBibitem{\clearfield{day}}
\AtEveryBibitem{\clearfield{issn}}
\DeclareFieldFormat[article]{volume}{\mkbibbold{#1}}
\addbibresource{Referencias.bib}


\begin{document}
	
	%------ Encabezado -------- %
	
	\begin{center}
		\begin{minipage}{3cm}
			\begin{center}
				\includegraphics[height=3.4cm]{Logo_UNAM (1)}
			\end{center}
		\end{minipage}\hfill
		\begin{minipage}{3cm}
			\begin{center}
				\includegraphics[height=3.4cm]{Logo_FC (1)}
			\end{center}
		\end{minipage}
	\end{center}
	
	\rule{17cm}{0.1mm}
	
	%------ Fin de encabezado -------- %
	\vspace{0.5cm}
	
	\hspace{10cm}{\raggedleft Ciudad Universitaria, 10 de abril de 2025.}
	
	\hspace{1cm}
	
	\vspace{0.5cm}
	
	A la Coordinación de Física Biomédica
	
	Facultad de Ciencias
	
	Universidad Nacional Autónoma de México\\
	
	Por medio de la presente, comunico a ustedes el plan de actividades para la titulación en la modalidad de tesis de la alumna Dana Larissa Luna González, con número de cuenta 421122680, quien desarrollará el proyecto titulado ``Cálculo numérico de propiedades ópticas de eritrocitos sanos y enfermos'' bajo mi asesoría en el grupo de investigación en Nanoplasmónica del Departamento de Física, Facultad de Ciencias, UNAM.\\
	
	El estudio de las propiedades ópticas de las células biológicas como los osteoblastos \cite{Osteoblastos}, los linfocitos \cite{Linfocitos}, y los eritrocitos \cite{Blood}, es de importancia para el área médica. A partir del análisis de las propiedades ópticas, se obtiene información sobre la composición y el estado morfológico celular, lo que tiene aplicaciones potenciales en el diagnóstico y la detección temprana de diversas enfermedades, incluyendo ciertos tipos de cáncer e infecciones virales \cite{Linfocitos}. En particular, el estudio de los eritrocitos tiene un papel importante en el diagnóstico de distintas formas de anemia, así como en el diseño de terapias ópticas emergentes, como el tratamiento de venas varicosas \cite{Blood}. \\
	
	En este contexto, los eritrocitos, los principales responsables de las propiedades ópticas de la sangre, juegan un papel clave. Factores como el hematocrito, la concentración de hemoglobina y el nivel de saturación de oxígeno influyen directamente en los mecanismos de absorción y esparcimiento de la luz de los eritrocitos \cite{Blood}, al modificar tanto su índice de refracción como su morfología celular. Por ello, el objetivo de esta tesis de licenciatura es estudiar las propiedades ópticas de eritrocitos sanos y enfermos, considerando variaciones en su índice de refracción y forma, para cuantificar los efectos en las secciones transversales de extinción, esparcimiento y absorción. Se propone utilizar el método de elementos finitos (FEM, por sus siglas en inglés) y comparar los resultados obtenidos con aquellos reportados en la literatura que emplean métodos alternativos.\\
	
	Durante el desarrollo de la tesis, la alumna estudiará el método de elementos finitos y las principales enfermedades que afectan a los eritrocitos. Posteriormente, se familiarizará con el uso del software SolidWorks para el diseño de las distintas morfologías celulares, las cuales serán implementadas en COMSOL Multiphysics con el fin de analizar casos con diferentes índices de refracción asociados a diversas patologías. Asimismo, se realizará la validación del modelo numérico mediante casos analíticos, como la teoría de Mie. Con base en las etapas mencionadas, se propone el siguiente cronograma de actividades:
	
	\begin{itemize} 
		\item \textbf{Abril – mayo:} Revisión de la literatura sobre las propiedades ópticas de eritrocitos sanos y enfermos, así como la clasificación de sus principales alteraciones morfológicas y fisiológicas.
		\item \textbf{Junio – julio:} Estudio del método de elementos finitos y familiarización con los programas COMSOL Multiphysics y SolidWorks. Paralelamente, se iniciará la redacción de los capítulos introductorios de la tesis escrita.
		
		\item \textbf{Agosto – septiembre:} Validación del modelo numérico mediante comparación con casos analíticos conocidos, como la teoría de Mie.
		
		\item \textbf{Septiembre – octubre:} Cálculo de las secciones transversales de absorción, extinción y esparcimiento para eritrocitos sanos y para al menos dos casos de eritrocitos patológicos. Análisis y contraste de los resultados obtenidos.
		
		\item \textbf{Octubre – diciembre:} Redacción del análisis de resultados y de las conclusiones finales de la tesis escrita.
	\end{itemize}
	
	
	
	\bigskip
	
	{\vspace{2.55cm}\begin{tabular} { c}
			\setlength{\tabcolsep}{15pt}
			\renewcommand{\arraystretch}{1}
			\noindent\rule{5.5cm}{0.4pt}\qquad \\
			
			\qquad  \textbf{Dana Larissa Luna González} \qquad \\
			\qquad Estudiante de Física Biomédica  \qquad \\ \qquad 
			No. de cuenta: 421122680\qquad \\  
			\qquad  Tel.: 776 101 4262 \qquad \\
			\qquad dana.larissalg@ciencias.unam.mx \qquad \\
			
		\end{tabular}
	}
	
	{\vspace{-2.85cm}\hspace{7cm}\begin{tabular} { c}
			\setlength{\tabcolsep}{15pt}
			\renewcommand{\arraystretch}{1}
			\noindent\rule{5.5cm}{0.4pt}\qquad \\
			
			\qquad  \textbf{Alejandro Reyes Coronado} \qquad \\
			\qquad Profesor Titular C de Tiempo Completo  \qquad \\  
			\qquad Departamento de Física, Facultad de Ciencias, UNAM\qquad \\ 
			\qquad  Tel.: (55) 5622 4968 \qquad \\
			\qquad coronado@ciencias.unam.mx \qquad \\
			
		\end{tabular}
		
	}
	
	
	
	
	
	
	
	
	
	
	
	
	
	
	
	
	
	
	
	
	
	
	
	
	
	
	
	%%----------------------------------------
	\printbibliography
	
\end{document}



