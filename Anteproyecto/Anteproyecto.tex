\documentclass[11pt,letterpaper]{article}
\usepackage[utf8]{inputenc}
\usepackage[spanish]{babel} 
\decimalpoint

%----- Configuración del estilo del documento------%
\usepackage{epsfig,graphicx}
\usepackage[left=2cm,right=2cm,top=1.8cm,bottom=2.3cm]{geometry}
\usepackage{fancyhdr}
\usepackage{tikz}
\usepackage{enumitem} %cambiar el estilo de la numeración de las listas
\pagestyle{fancy}
\fancyhf{}
\fancyfoot{}
\graphicspath{{Figuras/}}
\usepackage{hyperref}
\hypersetup{colorlinks=true,linkcolor=blue,citecolor=blue,filecolor=blue,urlcolor=magenta,}

\fancyfoot[RO,LE]{\thepage} % Custom footer text
\fancyheadoffset[RO,LE]{0.01\textwidth}

%------ Paquetes matemáticos básicos --------%
\usepackage{amsmath}
\usepackage{amssymb}
\usepackage{amsthm}

%------ Texto aleatorio ----- %

\usepackage{lipsum}


\usepackage[bibstyle = trad-abbrv,
citestyle = numeric-comp,
backend = bibtex, 
sorting = none, % Sort as they appear
backref=false]{biblatex} %style = trad-abbrv 
\AtEveryBibitem{\clearfield{urldate}}
\AtEveryBibitem{\clearfield{url}}
\AtEveryBibitem{\clearfield{isbn}}
\AtEveryBibitem{\clearfield{month}}
\AtEveryBibitem{\clearfield{day}}
\AtEveryBibitem{\clearfield{issn}}
\AtEveryBibitem{\clearfield{serie}}
\DeclareFieldFormat[article]{volume}{\mkbibbold{#1}}
\addbibresource{Tesis.bib}


\begin{document}
	
	%------ Encabezado -------- %
	
	\begin{center}
		\begin{minipage}{3cm}
			\begin{center}
				\includegraphics[height=3.4cm]{Logo_UNAM (1)}
			\end{center}
		\end{minipage}\hfill
		\begin{minipage}{3cm}
			\begin{center}
				\includegraphics[height=3.4cm]{Logo_FC (1)}
			\end{center}
		\end{minipage}
	\end{center}
	
	\rule{17cm}{0.1mm}
	
	%------ Fin de encabezado -------- %
	\vspace{0.5cm}
	
	\hspace{10cm}{\raggedleft Ciudad Universitaria, 21 de abril de 2025}
	
	\hspace{1cm}
	
	\vspace{0.5cm}
	
	\noindent\textsc{Universidad Nacional Autónoma de México}
	
	\noindent\textsc{Facultad de Ciencias
	}
	
	\noindent\textsc{Coordinación de Física Biomédica
	}
	
	\noindent\textsc{Comité Académico de Titulación
	}
	
	\noindent\textsc{PRESENTE}\\
	
	\noindent
	% 
	Por medio de la presente, comunico a ustedes el plan de actividades de la alumna Dana Larissa Luna González, con número de cuenta 421122680, para  iniciar el trámite de titulación en la modalidad de tesis. La alumna desarrollará el proyecto titulado ``Cálculo numérico de propiedades ópticas de eritrocitos sanos y enfermos'' bajo mi asesoría, en el grupo de Nanoplasmónica del Departamento de Física, Facultad de Ciencias, UNAM.\\
	
	El estudio de las propiedades ópticas de las células biológicas como los osteoblastos \cite{antunesOpticalPropertiesBone2019}, los linfocitos~\cite{yoonIdentificationNonactivatedLymphocytes2017} y los eritrocitos \cite{bosschaartLiteratureReviewNovel2014}, es de importancia para el área médica por sus potenciales aplicaciones en el diagnóstico y la detección temprana de enfermedades.  En particular, se ha reportado que los eritrocitos presentan alteraciones en su composición y morfología ante diversas formas de anemia \cite{bosschaartLiteratureReviewNovel2014}, lo que afecta directamente su respuesta óptica \cite{wriedtLightScatteringSingle2006,meinkeOpticalPropertiesPlatelets2007}.\\
	
	La respuesta óptica de los eritrocitos se modifica por factores como el hematocrito, la concentración de hemoglobina y el nivel de saturación de oxígeno \cite{bosschaartLiteratureReviewNovel2014}, los cuales modifican su función dieléctrica y su geometría, cambiando a su vez sus mecanismos de absorción y esparcimiento de luz \cite{bosschaartLiteratureReviewNovel2014}. Por lo tanto, el objetivo principal de la tesis de licenciatura de la alumna Dana Larissa es caracterizar las propiedades ópticas de eritrocitos tanto sanos como enfermos, considerando variaciones en su función dieléctrica y morfología. Para ello, se propone cuantificar la respuesta espectral de los eritrocitos mediante el cálculo de las secciones transversales de extinción, esparcimiento y absorción, empleando el método de elemento finito (FEM, por sus siglas en inglés).\\
	
	Durante el desarrollo del trabajo, la alumna abordará tres ejes temáticos principales. En el primer eje ahondará en las patologías más comunes que afectan a los eritrocitos. En el segundo eje, se familiarizará con la solución analítica conocida como teoría de Mie, así como con las relaciones de Kramers-Kronig~\cite{lucariniKramersKronigRelationsOptical2005} para determinar la función dieléctrica de los eritrocitos considerados. Finalmente, en el tercer eje, calculará numéricamente la repuesta óptica, esparcimiento y absorción de luz, para los eritrocitos con el FEM implementado en COMSOL Multiphysics\textsuperscript{\texttrademark} y geometrías diseñadas en SolidWorks. En el grupo de Nanoplasmónica contamos con las licencias para ambos software. Para cumplir con los objetivos planteados, el cronograma de actividades es el siguiente:
	
	\begin{itemize} 
		\item \textbf{Abril – mayo:} Revisión de la literatura sobre las propiedades ópticas de eritrocitos sanos y enfermos, así como la clasificación de sus principales alteraciones morfológicas y fisiológicas.
		\item \textbf{Junio – agosto:} Familiarización con los programas COMSOL Multiphysics\textsuperscript{\texttrademark} y SolidWorks. 
		
		\item \textbf{Septiembre - octubre:} Obtención de la función dieléctrica por medio de las relaciones de Kramers-Kronig a partir de datos reportados en la literatura.
		
		\item \textbf{Noviembre – enero:} Cálculo de las secciones transversales de absorción, extinción y esparcimiento para eritrocitos sanos y enfermos. 
		
		\item \textbf{Febrero – mayo:} Análisis de resultados e inicio de redacción de la tesis.
	\end{itemize}
	

	Para el trabajo escrito, se propone el siguiente índice
	\begin{itemize}
		\setlength\itemsep{0.05em}
		\item  Introducción: Justificación y antecedentes del trabajo
		\begin{enumerate}
			\item Esparcimiento de luz por partículas
			\begin{enumerate}[label=1.\arabic*]
				\item Fundamentos
				\item Secciones transversales
				\item Relaciones de Kramers-Kronig
			\end{enumerate}
			\item Eritrocitos y sus patologías
			\item Método de elemento finito
			\begin{enumerate}[label=3.\arabic*]
				\item Convergencia numérica 
			\end{enumerate}
			\item Resultados
			\begin{enumerate} [label=4.\arabic*]
				\item Propiedades ópticas de eritrocitos sanos
				\item  Contraste de las propiedades ópticas con eritrocitos enfermos
			\end{enumerate}
		\end{enumerate}	
		\item Conclusiones y trabajo a futuro	
	\end{itemize}
	%
	Atentamente,
	
	\vspace{1cm}
	{\hspace{0.7cm}\begin{tabular}{c}
		\includegraphics[height=2.cm]{firma}\\[-0.8cm] % Ajusta altura y espacio vertical aquí
		\rule{5.5cm}{0.4pt} \\[0.2cm]
		\textbf{Dana Larissa Luna González} \\
		Estudiante de Física Biomédica \\
		No. de cuenta: 421122680 \\
		Tel.: 776 101 4262 \\
		dana.larissalg@ciencias.unam.mx \\
	\end{tabular}}

	{\vspace{-3.1cm}\hspace{7.5cm}\begin{tabular} { c}
		\rule{5.5cm}{0.4pt} \\[0.2cm]
		\textbf{Dr. Alejandro Reyes Coronado} \\
		Profesor Titular C de Tiempo Completo \\
		Departamento de Física, Facultad de Ciencias, UNAM \\
		Tel.: (55) 5622 4968 \\
		coronado@ciencias.unam.mx \\
	\end{tabular}
	
	
	
	
	
	
	
	
	
	
	
	
	
	
	
	
	
	
	
	
	
	
	
	
	
	
	
	
	
	%%----------------------------------------
	\printbibliography
	
\end{document}



