\documentclass[11pt,letterpaper]{article}
\usepackage[utf8]{inputenc}
\usepackage[spanish]{babel} 
\decimalpoint

%----- Configuración del estilo del documento------%
\usepackage{epsfig,graphicx}
\usepackage[left=2cm,right=2cm,top=1.8cm,bottom=2.3cm]{geometry}
\usepackage{fancyhdr}
\pagestyle{fancy}
\fancyhf{}
\fancyfoot{}
\graphicspath{{Figuras/}}
\usepackage{hyperref}
\hypersetup{colorlinks=true,linkcolor=blue,citecolor=blue,filecolor=blue,urlcolor=magenta,}

\fancyfoot[RO,LE]{\thepage} % Custom footer text
\fancyheadoffset[RO,LE]{0.01\textwidth}

%------ Paquetes matemáticos básicos --------%
\usepackage{amsmath}
\usepackage{amssymb}
\usepackage{amsthm}

%------ Texto aleatorio ----- %

\usepackage{lipsum}


\usepackage[bibstyle = trad-abbrv,
citestyle = numeric-comp,
backend = bibtex, 
sorting = none, % Sort as they appear
backref=false]{biblatex} %style = trad-abbrv 
\AtEveryBibitem{\clearfield{urldate}}
\AtEveryBibitem{\clearfield{url}}
\AtEveryBibitem{\clearfield{isbn}}
\AtEveryBibitem{\clearfield{month}}
\AtEveryBibitem{\clearfield{day}}
\AtEveryBibitem{\clearfield{issn}}
\AtEveryBibitem{\clearfield{serie}}
\DeclareFieldFormat[article]{volume}{\mkbibbold{#1}}
\addbibresource{Tesis.bib}


\begin{document}
	
	%------ Encabezado -------- %
	
	\begin{center}
		\begin{minipage}{3cm}
			\begin{center}
				\includegraphics[height=3.4cm]{Logo_UNAM (1)}
			\end{center}
		\end{minipage}\hfill
		\begin{minipage}{3cm}
			\begin{center}
				\includegraphics[height=3.4cm]{Logo_FC (1)}
			\end{center}
		\end{minipage}
	\end{center}
	
	\rule{17cm}{0.1mm}
	
	%------ Fin de encabezado -------- %
	\vspace{0.5cm}
	
	\hspace{10cm}{\raggedleft Ciudad Universitaria, 22 de abril de 2025}
	
	\hspace{1cm}
	
	\vspace{0.5cm}
	
	\textsc{Universidad Nacional Autónoma de México}
	
	\textsc{Facultad de Ciencias
	}
	
	\textsc{Comité Académico de Titulación
	}
	
	\textsc{PRESENTE}\\
	
	\noindent
	% 
	Por medio de la presente, comunico a ustedes el plan de actividades de la alumna Dana Larissa Luna González, con número de cuenta 421122680, para  iniciar el trámite de titulación en la modalidad de tesis quien desarrollará el proyecto titulado ``Cálculo numérico de propiedades ópticas de eritrocitos sanos y enfermos'' bajo mi asesoría en el grupo de Nanoplasmónica adscrito al Departamento de Física, Facultad de Ciencias, UNAM.\\
	
	El estudio de las propiedades ópticas de las células biológicas como los osteoblastos \cite{antunesOpticalPropertiesBone2019}, los linfocitos \cite{yoonIdentificationNonactivatedLymphocytes2017}, y los eritrocitos \cite{bosschaartLiteratureReviewNovel2014}, es de importancia para el área médica por sus potenciales aplicaciones en el diagnóstico y la detección temprana de enfermedades.  En particular, se ha reportado que  los eritocitos  presentan un cambio en su composición y estado morfológico  ante distitntas formas de anemia \cite{bosschaartLiteratureReviewNovel2014}. En este contexto, los eritrocitos, los principales responsables de las propiedades ópticas de la sangre, juegan un papel clave.\\
	
	La respuesta óptica de los eritrocitos se modifica por factores como el hematocrito, la concentración de hemoglobina y el nivel de saturación de oxígeno, los cuales modifican directamente los mecanismos de absorción y esparcimiento de la luz de los eritrocitos \cite{bosschaartLiteratureReviewNovel2014}, al modificar tanto su índice de refracción como su morfología celular. Por ello, el objetivo de esta tesis de licenciatura es caracterizar las propiedades ópticas de eritrocitos tanto sanos como enfermos, considerando variaciones en su índice de refracción y morfología, para cuantificar la respuesta espectral de los eritocitos mediante el cálculo de las secciones transversales de extinción, esparcimiento y absorción. Para el cálculo de las secciones transversales, se propone utilizar el método de elemento finito (FEM, por sus siglas en inglés) y comparar los resultados obtenidos con los obtenidos con otras metodologías \cite{ergulComputationalStudyScattering2010,wriedtLightScatteringSingle2006}.\\
	
	Durante el desarrollo de la tesis, la alumna
	incursionará en temas de interés médico sobre eritrositos, en teorías analíticas de esparcimiento de luz y métodos numéricos para la solución de la ecuación de Helmholtz. En el primer rubro se estudiarán las enfermedades más comunes que afectan eritrositos. En el segundo, la solucioón analítica de teoría de Mie, y el uso de las relaciones de Kramers Kronig \cite{lucariniKramersKronigRelationsOptical2005} para determinar la respuesta óptica de los eritrocitos considerados. Finalmente, la alumna se familirizará con el FEM implementado en el software comercial COMSOL Multiphysics con el fin de analizar casos con diferentes índices de refracción asociados a diversas patologías y en la construcción de  geometrías complejas mediante SolidWorks para realizar los cálculos de eritrositos. Ambos softwares requieren de una licencia ya en uso por el grupo de Nanoplasmónica.\\
	
	Con base en las etapas mencionadas, se propone el siguiente cronograma de actividades:
	
	\begin{itemize} 
		\item \textbf{Abril – mayo:} Revisión de la literatura sobre las propiedades ópticas de eritrocitos sanos y enfermos, así como la clasificación de sus principales alteraciones morfológicas y fisiológicas.
		\item \textbf{Junio – julio:} Estudio del método de elementos finitos y familiarización con los programas COMSOL Multiphysics y SolidWorks. Paralelamente, se iniciará la redacción de los capítulos introductorios de la tesis escrita.
		
		\item \textbf{Agosto – septiembre:} Validación del modelo numérico mediante comparación con casos analíticos conocidos, como la teoría de Mie.
		
		\item \textbf{Septiembre – octubre:} Cálculo de las secciones transversales de absorción, extinción y esparcimiento para eritrocitos sanos y para al menos dos casos de eritrocitos patológicos. Análisis y contraste de los resultados obtenidos.
		
		\item \textbf{Octubre – diciembre:} Redacción del análisis de resultados y de las conclusiones finales de la tesis escrita.
	\end{itemize}
	

	Para el trabajo escrito, se propone el siguiente índice:
	\begin{itemize}
		\item [] Agradecimientos
		\item [] Resumen
		\item [] Introducción: Justificación y antecedentes del trabajo
		\begin{enumerate}
			\item Esparcimiento de ondas EM
			\begin{enumerate}
				\item Fundamentos
				\item FEM
				\item Kramers-Kronig
			\end{enumerate}
			\item Eritrocitos y enfermedades sanguíneas
			\begin{enumerate}
				\item Funciones dieléctricas
			\end{enumerate}
			\item Resultados
			\begin{enumerate}
				\item Convergencia
				\item Propiedades de un eritrocito snao
				\item  Contraste con eritrocitos enfermos
			\end{enumerate}
			\item Conclusiones
			\item Trabajo a futuro	
		\end{enumerate}
		\item[] Apéndice
	\end{itemize}
	
	Atentamente,
	
	
	\bigskip
	
	{\vspace{2.55cm}\begin{tabular} { c}
			\setlength{\tabcolsep}{15pt}
			\renewcommand{\arraystretch}{1}
			\noindent\rule{5.5cm}{0.4pt}\qquad \\
			
			\qquad  \textbf{Dana Larissa Luna González} \qquad \\
			\qquad Estudiante de Física Biomédica  \qquad \\ \qquad 
			No. de cuenta: 421122680\qquad \\  
			\qquad  Tel.: 776 101 4262 \qquad \\
			\qquad dana.larissalg@ciencias.unam.mx \qquad \\
			
		\end{tabular}
	}
	
	{\vspace{-2.85cm}\hspace{7cm}\begin{tabular} { c}
			\setlength{\tabcolsep}{15pt}
			\renewcommand{\arraystretch}{1}
			\noindent\rule{5.5cm}{0.4pt}\qquad \\
			
			\qquad  \textbf{Dr. Alejandro Reyes Coronado} \qquad \\
			\qquad Profesor Titular C de Tiempo Completo  \qquad \\  
			\qquad Departamento de Física, Facultad de Ciencias, UNAM\qquad \\ 
			\qquad  Tel.: (55) 5622 4968 \qquad \\
			\qquad coronado@ciencias.unam.mx \qquad \\
			
		\end{tabular}
		
	}
	
	
	
	
	
	
	
	
	
	
	
	
	
	
	
	
	
	
	
	
	
	
	
	
	
	
	
	%%----------------------------------------
	\printbibliography
	
\end{document}



