\documentclass[9pt,letterpaper]{article}
\usepackage[utf8]{inputenc}
\usepackage[spanish]{babel} 
\decimalpoint

%----- Configuración del estilo del documento------%
\usepackage{epsfig,graphicx}
\usepackage[left=2cm,right=2cm,top=1.8cm,bottom=2.3cm]{geometry}
\usepackage{fancyhdr}
\pagestyle{fancy}
\fancyhf{}
\fancyfoot{}
\graphicspath{{Figuras/}}
\usepackage{hyperref}
\hypersetup{colorlinks=true,linkcolor=blue,citecolor=blue,filecolor=blue,urlcolor=magenta,}

\fancyfoot[RO,LE]{\thepage} % Custom footer text
\fancyheadoffset[RO,LE]{0.01\textwidth}

%------ Paquetes matemáticos básicos --------%
\usepackage{amsmath}
\usepackage{amssymb}
\usepackage{amsthm}

%------ Texto aleatorio ----- %

\usepackage{lipsum}


\usepackage[bibstyle = trad-abbrv,
citestyle = numeric-comp,
backend = bibtex, 
sorting = none, % Sort as they appear
backref=true]{biblatex} %style = trad-abbrv 
\AtEveryBibitem{\clearfield{urldate}}
\AtEveryBibitem{\clearfield{url}}
\AtEveryBibitem{\clearfield{isbn}}
\AtEveryBibitem{\clearfield{month}}
\AtEveryBibitem{\clearfield{day}}
\AtEveryBibitem{\clearfield{issn}}
\DeclareFieldFormat[article]{volume}{\mkbibbold{#1}}
\addbibresource{Referencias.bib}


\begin{document}
	
	%------ Encabezado -------- %
	
	\begin{center}
		\begin{minipage}{3cm}
			\begin{center}
				\includegraphics[height=3.4cm]{Logo_UNAM (1)}
			\end{center}
		\end{minipage}\hfill
		\begin{minipage}{3cm}
			\begin{center}
				\includegraphics[height=3.4cm]{Logo_FC (1)}
			\end{center}
		\end{minipage}
	\end{center}
	
	\rule{17cm}{0.1mm}
	
	%------ Fin de encabezado -------- %
	\hspace{0cm}
	
	\parbox{\textwidth}{\raggedleft Ciudad Universitaria, 10 de abril de 2025.}
	
	\hspace{1cm}
	
	\vspace{0.5cm}
	
	A la Coordinación de Física Biomédica
	
	Facultad de Ciencias
	
	Universidad Nacional Autónoma de México\\
	
	Por medio de la presente, comunico a ustedes el plan de actividades para la titulación en modalidad de tesis de la alumna Dana Larissa Luna González, con número de cuenta 421122680, quien realizará su tesis con título ``Propiedades'' bajo mi asesoría en mi grupo de investigación en Nanoplasmónica, Departamento de Física, Facultad de Ciencias, UNAM.\\
	
	El estudio de las propiedades ópticas de las células biológicas como los osteoblastos \cite{Osteoblastos}, los linfocitos \cite{Linfocitos}, y los eritrocitos \cite{Blood}, es de importancia para el área médica. A partir de las propiedades ópticas de las células, se obtiene información de su composición y estado morfológico, lo cual tiene potenciales aplicaciones en el diagnóstico y la detección temprana de diversas enfermedades, incluidos cánceres e infecciones virales \cite{Linfocitos}. En particular, el estudio de los eritrocitos tiene un papel importante en el diagnóstico de la anemia en sus diferentes tipos y el diseño de nuevas terapias ópticas, como el tratamiento de las venas varicosas \cite{Blood}. \\
	
	Los eritrocitos sanos presentan forma de discoides cóncavos con longitudes de entre 4 a 9 $\mu$m de diámetro. Estos no poseen núcleo, por lo que pueden modelarse como un objeto homogéneo \cite{Cassini}. Debido a su forma, para simplificar el proceso de modelado, se han empleado diferentes opciones como los óvalos de Cassini \cite{Cassini} o funciones en términos de coordenadas esféricas \cite{Esferico}. Dado que los glóbulos rojos son los principales responsables de las propiedades ópticas de la sangre, su hematocrito, la concentración de hemoglobina y la saturación de oxígeno influyen directamente en las propiedades de absorción y dispersión de la sangre a través de cambios en su índice de refracción y en su forma. El objetivo de esta tesis de licenciatura es el de estudiar propiedades ópticas de eritrocitos sanos y enfermos mediante el método de elementos finitos (FEM por sus siglas en inglés) y contrastar los resultados con resultados reportados en la literatura donde se emplearon técnicas alternativas.\\
	
	
	 El proyecto lo realicé bajo la supervisión del Dr. Alejandro Reyes Coronado, en el Departamento de Física, cubículo 407 de la Facultad de Ciencias de la UNAM. Realicé las actividades en el periodo comprendido entre el 23 de febrero de 2024 al 23 de septiembre del 2024. \\
	
	
	El estudio de las propiedades ópticas de células biológicas, como los osteoblastos \cite{Osteoblastos}, los linfocitos \cite{Linfocitos} y los eritrocitos \cite{Blood}, es fundamental para aplicaciones médicas, incluyendo el diagnóstico de enfermedades y el desarrollo de terapias ópticas. En particular, los eritrocitos, debido a su forma discoide cóncava y la ausencia de núcleo, pueden modelarse como estructuras homogéneas \cite{Cassini}. Para su análisis, se han utilizado modelos como óvalos de Cassini \cite{Cassini} o funciones en coordenadas esféricas \cite{Esferico}. El objetivo de esta tesis es el de estudiar propiedades ópticas de eritrocitos sanos y enfermos mediante el método de elementos finitos (FEM por sus siglas en inglés) y contrastar los resultados con resultados reportados en la literatura donde se emplearon técnicas alternativas. Para corroborar la convergencia numérica, se compará el resultado con Comsol y la solución analítica en el régimen cuasiestático de las propiedades ópticas de un elipsoide de con una función dieléctrica tipo Drude y un material dieléctrico.\\
	
	Durante el tiempo en el que se desarrollará la tesis, la alumna estudiará el método de elementos finitos y las diferentes enfermedades de eritrocitos. Posteriormente, se familiarizará con el uso de software SolidWorks para el diseño de las diferentes formas de eritrocitos y se implementarán las diferentes geometrías en COMSOL y se estudiarpan casos con diferentes índices de refracción segun diferentes enfermedades. Se realizará la validación del modelo numérico con casos analíticos como el de teoría de Mie. Con ello, se propone el siguiente cronograma de actividades:
	\begin{itemize}
		\item \textbf{De abril a mayo:} La alumna realizará una revisión de la literatura sobre propiedades ópticas de eritrocitos sanos y enfermos, así como la clasificación de sus diferentes enfermedades. 
		\item \textbf{De junio a julio:} La alumna estudiará el método de elementos finitos y se familiarizará con el uso de COMSOL y SolidWorks. De manera paralela, redactará los capítulos iniciales de la tesis escrita.
		\item \textbf{De agosto a septiembre:} La alumna validará el modelo numérico con casos analíticos como el de teoría de Mie.
		\item \textbf{De septiembre a octubre:} La alumna realizará el cálculo de las secciones transversales de absorción, extinción y esparcimiento para eritrocitos sanos y para dos casos de eritrocitos enfermos y contrastará los resultados.
		\item \textbf{De octubre a diciembre:} La alumna redactará los resultados y conclusiones obtenidos.
	\end{itemize}
	
	\bigskip
	
	{\vspace{2.55cm}\begin{tabular} { c}
			\setlength{\tabcolsep}{15pt}
			\renewcommand{\arraystretch}{1}
			\noindent\rule{5.5cm}{0.4pt}\qquad \\
			
			\qquad  \textbf{Dana Larissa Luna González} \qquad \\
			\qquad Estudiante de Física Biomédica  \qquad \\ \qquad 
			No. de cuenta: 421122680\qquad \\  
			\qquad  Tel.: 776 101 4262 \qquad \\
			\qquad dana.larissalg@ciencias.unam.mx \qquad \\
			
		\end{tabular}
	}
	
	{\vspace{-2.53cm}\hspace{7cm}\begin{tabular} { c}
			\setlength{\tabcolsep}{15pt}
			\renewcommand{\arraystretch}{1}
			\noindent\rule{5.5cm}{0.4pt}\qquad \\
			
			\qquad  \textbf{Alejandro Reyes Coronado} \qquad \\
			\qquad Profesor Titular C de Tiempo Completo  \qquad \\  
			\qquad Departamento de Física, Facultad de Ciencias, UNAM\qquad \\ 
			\qquad  Tel.: (55) 5622 4968 \qquad \\
			\qquad coronado@ciencias.unam.mx \qquad \\
			
		\end{tabular}
		
	}
	
	
	
	
	
	
	
	
	
	
	
	
	
	
	
	
	
	
	
	
	
	
	
	
	
	
	
	%%----------------------------------------
	\printbibliography
	
\end{document}



