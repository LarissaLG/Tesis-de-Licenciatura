% !TeX root = ../tesis.tex


\section{Relaciones de Kramers-Kronig}
\label{section:yth}

El estudio experimental de las propiedades ópticas lineales no siempre permite obtener todas las propiedades ópticas de una muestra. Afortunadamente, el principio de causalidad en la respuesta óptica de cualquier material permite extraer información a partir de los datos experimentales medidos. En el caso de un sistema de esparcimiento, el principio de causalidad impone que ninguna onda esparcida puede existir antes de que la onda incidente haya alcanzado alguna parte de la partícula esparcidora, cuyo tamaño se supone finito.

Las relaciones K-K describen una conexión entre las partes reales e imaginarias de funciones ópticas lineales complejas que describen fenómenos de interacción luz-materia, como la susceptibilidad, la función dieléctrica, el índice de refracción y la reflectividad. Las partes reales e imaginarias no son totalmente independientes, sino que están conectadas por una forma especial de transformadas de Hilbert, denominadas relaciones K-K.  Además, al aplicar las relaciones K-K, es posible realizar la llamada inversión de datos ópticos, es decir, adquirir conocimiento sobre los fenómenos dispersivos mediante mediciones de los fenómenos absortivos en todo el espectro (por ejemplo, con espectroscopia de transmisión) o viceversa. 

En la espectroscopia óptica lineal, el análisis K-K tiene dos funciones típicas, según si la medición se basa en la transmisión o la reflexión de la luz. En el primer caso, se suele medir la parte imaginaria y la parte real se obtiene mediante una relación K-K, mientras que en el segundo, se mide la intensidad y se calcula la fase mediante la relación K-K adecuada.
Sin embargo, al utilizar un elipsómetro, es posible obtener información experimental a partir de la función compleja obtenida mediante la medición. En tal caso, las relaciones K-K pueden utilizarse para comprobar la autoconsistencia de las partes reales e imaginarias medidas de los datos.

Las relaciones de Kramers-Kronig, se obtienen a partir de la Ec. \eqref{eq:TF_D_final} 



