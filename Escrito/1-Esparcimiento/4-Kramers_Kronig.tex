% !TeX root = ../tesis.tex


\section{Relaciones de Kramers-Kronig}
\label{section:yth}

El estudio experimental de las propiedades ópticas lineales no siempre permite determinar por completo las características ópticas de una muestra. No obstante, el principio de causalidad en la respuesta óptica de cualquier material permite extraer información adicional a partir de los datos experimentales.  A partir de la aplicación del principio de causalidad y de la analiticidad de una función óptica lineal compleja, como la susceptibilidad, la función dieléctrica, el índice de refracción o la reflectividad, se deducen las relaciones de Kramers-Kronig (KK), las cuales describen la conexión entre las partes real e imaginaria de una función óptica lineal compleja. Para deducir las relaciones de KK se considera un material homogéneo e isótropo caracterizado por una función óptica lineal compleja, que en este caso será la función dieléctrica compleja, así como un espacio de frecuencias complejo.

La transformada de Fourier temporal para $\vb{D}$ está dada por~\cite{jacksonClassicalElectrodynamics2021a}
%
\begin{equation}
	\vb{D}(\vb{r},\omega)=\int_{-\infty}^{\infty}\vb{D}(\vb{r},t')e^{i\omega t'}\text{d}t',\label{eq:TFI}
\end{equation}
%
mientras que la transformada de Fourier inversa es
%
\begin{equation}
	\vb{D}(\vb{r},t)=\frac{1}{2\pi}\int_{-\infty}^{\infty}\vb{D}(\vb{r},\omega)e^{-i\omega t}\text{d}\omega.\label{eq:TF}
\end{equation}
%
Al sustituir la Ec. \eqref{eq:d3} en la Ec. \eqref{eq:TF}, se obtiene
%
\begin{equation}
	\vb{D}(\vb{r},t)=\frac{1}{2\pi}\int_{-\infty}^{\infty}\varepsilon(\omega)\vb{E}(\vb{r},\omega)e^{-i\omega t}\text{d}\omega,\label{eq:TF_D_intermedia_1}
\end{equation}
%
y al sustituir la transformada de Fourier de $\vb{E}(\vb{r},t)$, se tiene que
%
\begin{equation}
	\vb{D}(\vb{r},t)=\frac{1}{2\pi}\int_{-\infty}^{\infty}\varepsilon(\omega)e^{-i\omega t}\text{d}\omega\int_{-\infty}^{\infty}e^{i\omega t'}\vb{E}(\vb{r},t')\text{d}t'.\label{eq:TF_D_intermedia_2}
\end{equation}
%
Considerando que los órdenes de integración se pueden intercambiar, dado que la integración se realiza en el mismo intervalo ($-\infty$, $\infty$), la Ec. \eqref{eq:TF_D_intermedia_2} se reescribe como
%
\begin{tcolorbox}[ams align]
	\vb{D}(\vb{r},t)=\frac{1}{2\pi}\int_{-\infty}^{\infty}\int_{-\infty}^{\infty}\varepsilon(\omega)\vb{E}(\vb{r},t') e^{i\omega (t-t')} \text{d}\omega \text{d}t'.\label{eq:TF_D_intermedia_3} 
\end{tcolorbox}
%
\noindent Al realizar el cambio de variable $\tau = t - t'$ y emplear la función constante\footnote{$\delta(\tau)$ es la función delta de Dirac, que cumple con $\int_{-\infty}^{\infty}f(\tau)\delta(\tau)\text{d}\tau=f(0)$ \cite{arfkenMathematicalMethodsPhysicists2011a}.}
\begin{equation}
	\mathcal{F}[1]=\int_{-\infty}^{\infty}e^{-i\omega\tau}=\delta(\tau),
\end{equation}
se reescribe al campo eléctrico como
\begin{equation}
	\vb{E}(\vb{r},t)=\int_{-\infty}^{\infty}\vb{E}(\vb{r},t-\tau)\delta(\tau)=\frac{1}{2\pi}\int_{-\infty}^{\infty}\int_{-\infty}^{\infty}\vb{E}(\vb{r},t-\tau)e^{-i\omega\tau}\text{d}\omega\text{d}\tau.
\end{equation}
%
De forma que al sumar $\vb{E}(\vb{r},t')-\vb{E}(\vb{r},t')$ y multiplicar por $\varepsilon_0/\varepsilon_0$ a la Ec. \eqref{eq:TF_D_intermedia_2}, simplificando se obtiene
%
\begin{equation}
	\vb{D}(\vb{r},t)=\varepsilon_0\left[\vb{E}(\vb{r},t)+\int_{-\infty}^{\infty}G(\tau)\vb{E}(\vb{r},t-\tau)\text{d}\tau\right],\label{eq:TF_D_final} 
\end{equation}
%
\noindent donde $G(\tau)$ es \cite{jacksonClassicalElectrodynamics2021a}
%
\begin{equation}
	G(\tau)=\frac{1}{2\pi}\int_{-\infty}^{\infty}\left[\frac{\varepsilon(\omega)}{\varepsilon_0}-1\right]e^{-i\omega\tau}\text{d}\omega,
	\label{eq:G_tdependent} 
\end{equation}
%
\noindent con $\varepsilon(\omega)/\varepsilon_0$ la permitividad eléctrica relativa. Al aplicar la transformada de Fourier a la Ec.~\eqref{eq:G_tdependent}, se obtiene una expresión para la permitividad eléctrica relativa
%
\begin{equation}
	\frac{\varepsilon(\omega)}{\varepsilon_0}=1+\int_{-\infty}^{\infty}G(\tau) e^{-i\omega\tau}\text{d}\tau.
	\label{eq:epsrelativa} 
\end{equation}
%

\noindent Las Ecs. \eqref{eq:TF_D_final} y \eqref{eq:G_tdependent} muestran que el campo de desplazamiento eléctrico al tiempo $t$ depende del campo eléctrico en todos los demás tiempos $t'$. A esta característica, se le conoce como la no localidad temporal entre $\vb{D}$ y $\vb{E}$ \cite{jacksonClassicalElectrodynamics2021a}. 

De modo que las Ecs. \eqref{eq:TF_D_final} y \eqref{eq:epsrelativa} sean causales, es necesario imponer condiciones que garanticen que la Ec. \eqref{eq:G_tdependent} se anule para $\tau<0$, lo que se traduce en que al tiempo $t$, únicamente valores del campo eléctrico previos a ese tiempo determinan el vector de desplazamiento eléctrico~\cite{jacksonClassicalElectrodynamics2021a}. De esta forma, las Ecs. \eqref{eq:TF_D_final} y \eqref{eq:epsrelativa} se reescriben como
%
\begin{subequations}%
	\begin{tcolorbox}[
		ams align, breakable]
		\vb{D}(\vb{r},t)&=\varepsilon_0\left[\vb{E}(\vb{r},t)+\int_{0}^{\infty}G(\tau)\vb{E}(\vb{r},t-\tau)\text{d}\tau\right],\\ \label{seq:D_final}
		\frac{\varepsilon(\omega)}{\varepsilon_0}-1&=\int_{0}^{\infty}G(\tau) e^{-i\omega\tau}\text{d}\tau.\label{seq:G}
	\end{tcolorbox}
\end{subequations}\vspace*{1em}
%
Dado que $\vb{E}$ y $\vb{D}$ son funciones reales, $G(\tau)$ también lo es. Si se considera a la Ec. \eqref{seq:G} como una representación de $	\varepsilon(\omega)/\varepsilon_0$ en el plano complejo, entonces la Ec. \eqref{seq:G} es analítica en el semiplano superior complejo siempre que $G(\tau)$ sea finita para toda $\tau$. De modo que la analiticidad se extienda sobre el eje real,  es necesario imponer la condición de que $G(\tau)\rightarrow 0$ si $\tau\rightarrow \infty$ \cite{jacksonClassicalElectrodynamics2021a} \footnote{Esto es únicamente cierto para dieléctricos, pues para conductores se tiene que $G(\tau)\rightarrow \sigma/\epsilon_0$ cuando $\tau\rightarrow \infty$~\cite{jacksonClassicalElectrodynamics2021a}. } .


Como $	\varepsilon(\omega)/\varepsilon_0-1$ es una función analítica en el semiplano superior complejo incluyendo el eje real, entonces $	[\varepsilon(\omega)/\varepsilon_0-1]/(\omega-\omega_0)$ también lo es, excepto en la singularidad de $\omega=\omega_0$. De acuerdo con el teorema integral de Cauchy\footnote{El teorema integral de Cauchy establece que si una función $f(z)$ es analítica en un contorno cerrado $C$ y dentro de la región interior delimitada por $C$, entonces $\oint_C f(z)/(z-z_0)\dd{z}= 0$ \cite{arfkenMathematicalMethodsPhysicists2011a}.} se tendrá entonces que para un contorno cerrado $C$ que no encierre a $\omega_0$ \cite{arfkenMathematicalMethodsPhysicists2011a}
%
\begin{equation}
	\oint_C \frac{[\varepsilon(\omega)/\varepsilon_0-1]}{\omega-\omega_0}\dd{\omega}= 0.
	\label{eq:Cauchy}
\end{equation}
%
Para determinar el contorno apropiado de modo que se pueda emplear la Ec. \eqref{eq:Cauchy}, se considera a $\omega_0$ un punto sobre el eje real y a $C$ la unión de cuatro curvas con representaciones paramétricas dadas por
%
\begin{align*}
	C_1&: \; \omega=\Omega, & (-A\leq \Omega \leq \omega-a)\\
	C_2&: \; \omega=\omega_0-ae^{-i\Omega}, & (0\leq\Omega\leq\pi)\\
	C_3&: \;\omega=\Omega, & (\omeg\;a+a\leq\omega\leq A)\\
	C_4&: \;\omega=Ae^{i\Omega}. & (0\leq\Omega\leq \pi)	
\end{align*}
%
Entonces, la Ec. \eqref{eq:Cauchy} se reescribe como
\begin{align*}
	\int_A^{\omega-a}\frac{[\varepsilon(\Omega)/\varepsilon_0-1]}{\Omega-\omega_0}\dd{\Omega}&+	\int_{\omega+a}^{A}\frac{[\varepsilon(\Omega)/\varepsilon_0-1])}{\Omega-\omega_0}\dd{\Omega}+\\
	&+	\int_0^{\pi}\frac{iA\;e^{i\Omega}\; [\varepsilon(A\;e^{i\Omega})/\varepsilon_0-1]}{A\;e^{i\Omega}-\omega_0}\dd{\Omega}-\int_0^{\pi}i\;[\varepsilon(\omega_0-ae^{i\Omega})/\varepsilon_0-1]\dd{\Omega}=0.
\end{align*}
Al integrar por partes la Ec. \eqref{seq:G}, se obtiene que 
%
\begin{equation*}
	\frac{\varepsilon(\omega)}{\varepsilon_0}-1=\frac{i G(0)}{\omega}-\frac{ G'(0)}{\omega^2}+\frac{i G''(0)}{\omega^3}+...,
\end{equation*}
%
donde se empleó que $G(\tau)\rightarrow 0$ si $\tau\rightarrow \infty$. Además, dado que $G(\tau)$ es causal, $G(0)=0$, por lo que el primer término no contribuye y los demás términos decaen muy rápido. Con ello, por el lema de Jordan, la integral sobre $C_4$ se desvanece conforme $A$ tiende a $\infty$ si $\lim_{\omega'\to\infty} \varepsilon(\omega')=0$. Al hacer $a\to 0$, la integral sobre $C_3$ es 0. Por otro lado, las integrales sobre $C_1$ y $C_3$ equivalen a 
%
\begin{equation}
	\text{PV}\int_{-\infty}^{\infty}\frac{[\varepsilon(\Omega)/\varepsilon_0-1]}{\omega'-\omega_0} \dd{\Omega}=i\pi [\varepsilon(\omega)/\varepsilon_0-1]
\end{equation}
%
donde PV denota el valor principal de la integral. Al separar sus contribuciones en parte real e imaginaria se obtienen











