% !TeX root = ../tesis.tex

\chapter{Esparcimiento de luz por partículas}
\label{chapter:theory}

\vspace*{7em}

It is recommended to write a summary about the contents of this chapter as an introduction to them. 


\section{Fundamentos}
\label{section:basics}

La electrodinámica clásica es descrita mediante la fuerza de Lorentz y las ecuaciones de Maxwell \cite{griffithsIntroductionElectrodynamics2023}. En el sistema internacional de unidades, las ecuaciones de Maxwell en su forma diferencial están dadas por \cite{griffithsIntroductionElectrodynamics2023}
%
	\begin{subequations} \label{eqs:Maxwell}
	\begin{tcolorbox}[
	ams align, breakable]
	\nabla \cdot\vb{E} &= \frac{\rho_{tot}}{\varepsilon_0}, &\mbox{(Ley de Gauss eléctrica)}  
	\label{seq:GE} \\
	\nabla \cdot\vb{B} &= 0,						&\mbox{(Ley de Gauss magnética)}   
	\label{seq:GM} \\
	\nabla \times\vb{E} &= -\pdv{\vb{B}}{t}, 	&\mbox{(Ley de Faraday-Lenz)}		
	\label{seq:FL}\\
	\nabla \times\vb{B} &= \mu_0 \vb{J}_{tot} +\varepsilon_0\mu_0 \pdv{\vb{E}}{t}, &
	\mbox{(Ley de Ampère-Maxwell)} \label{seq:AM}
	\end{tcolorbox}\end{subequations}\noindent
%
donde $\vb{E}$ representa al campo eléctrico y $\vb{B}$ al campo magnético; $\rho_{tot}$ representa a la densidad de carga volumétrica y $\vb{J}_{tot}$ a la densidad de corriente volumétrica; $\epsilon_0$ a la permitividad eléctrica en el vacío y $\mu_0$ a la permeabilidad magnética en el vacío. Las ecuaciones de Maxwell pueden desacoplarse para obtener ecuaciones de segundo orden separadas para $\vb{E}$ y $\vb{B}$. En particular, al considerar un medio sin fuentes externas, es decir, $\rho_{tot}=0$ y $\vb{J}_{tot}=0$, y emplear la transformada de Fourier,  se obtiene que los campos electromagnéticos satisfacen la ecuación de onda vectorial~\cite{jacksonClassicalElectrodynamics2021}

	\begin{subequations}%
	\eqhalf{\nabla^2\vb{E} + k^2 \vb{E}=\vb{0},}%
	\eqhalf{\nabla^2\vb{B} + k^2 \vb{B}=\vb{0}.}\label{eq:Helmholtz}%
	\end{subequations}\vspace*{-1em}

\noindent Una de las posibles soluciones son las ondas planas expresadas como 

	\begin{subequations}%
	\eqhalf{\vb{E}(\vb{r},t) =\vb{E_0}e^{i(\vb{k}\cdot\vb{r} -\omega t)},}%
	\eqhalf{\vb{B}(\vb{r}, t) =\vb{B_0}e^{i(\vb{k}\cdot\vb{r} -\omega t)}},	
	\label{eqs:ondasPlanas}\end{subequations}\vspace*{-1em}
		
\noindent donde $E_0$ y $B_0$ corresponden a las amplitudes de los campos, $\omega$ la frecuencia angular de la onda y $\vb{k}$ el número de onda. Para que se satisfagan las Ecs. \eqref{eqs:ondasPlanas}, se tiene que cumplir la relación de dispersión dada por $k=\sqrt{\mu\epsilon}\;\omega$, donde $\epsilon$ y $\mu$ corresponden a la permitividad eléctrica y la permeabilidad magnética del medio, respectivamente y que son en general, funciones complejas de $\omega$ \cite{jacksonClassicalElectrodynamics2021}. La relación de dispersión se puede reescribir en términos del índice de refracción del material dado por
%
	\begin{tcolorbox}[ams align]
				n(\omega) = \sqrt{\dfrac{\mu\varepsilon(\omega)}{\varepsilon_0 \mu_0}}.
			\label{eq:indice} 
		\end{tcolorbox}

%	
\noindent De forma que se obtiene,
%

%
\begin{tcolorbox}[ams align]
	k(\omega) = \sqrt{\dfrac{\omega\;c}{n(\omega)}},
	\label{eq:indice} 
\end{tcolorbox}

\noindent donde $c=1/\sqrt{\epsilon_0\mu_0}$ es la velocidad de la luz en el vacío.

La energía por unidad de tiempo, por unidad de área, transportada por los campos está dada por el vector de Poynting \cite{griffithsIntroductionElectrodynamics2023}
\begin{tcolorbox}[ams align]
	\vb{S}=\frac{1}{\mu_0}(\vb{E}\times\vb{B}),
	\label{eq:indice} 
\end{tcolorbox}


