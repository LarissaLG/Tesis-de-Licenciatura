% !TeX root = ../tesis.tex

\chapter{Esparcimiento de luz por partículas}
\label{chapter:theory}

\vspace*{7em}

En este capítulo


\section{Fundamentos}
\label{section:basics}
Todos los fenómenos electromagnéticos tienen su origen en una única interacción fundamental: la fuerza de Lorentz \cite{zangwillModernElectrodynamics2013}. Esta fuerza describe cómo actúa un campo electromagnético sobre una partícula cargada en movimiento. Si una partícula de carga \( q \) se desplaza con velocidad \( \vb{v} \) en presencia de un campo eléctrico \( \vb{E} \) y un campo magnético \( \vb{B} \), la fuerza que experimenta está dada por \cite{zangwillModernElectrodynamics2013}
%
\begin{equation}
	\vb{F}=q(\vb{E}+\vb{v}\times\vb{B}).
	\label{eq:lorentzforce} 
\end{equation}
%
La fuerza de Lorentz junto con las ecuaciones de Maxwell, describen a la electrodinámica clásica, que se centra en el origen y el comportamiento de los campos $\vb{E}$ y $\vb{B}$ \cite{zangwillModernElectrodynamics2013}. En unidades del Sistema Internacional, dichas ecuaciones se expresan en forma diferencial como:
\cite{griffithsIntroductionElectrodynamics2023b}
%
	\begin{subequations} \label{eqs:Maxwell}
	\begin{tcolorbox}[
	ams align, breakable]
	\nabla \cdot\vb{E} &= \frac{\rho_{\text{tot}}}{\varepsilon_0}, &\mbox{(Ley de Gauss eléctrica)}  
	\label{seq:GE} \\
	\nabla \cdot\vb{B} &= 0,						&\mbox{(Ley de Gauss magnética)}   
	\label{seq:GM} \\
	\nabla \times\vb{E} &= -\pdv{\vb{B}}{t}, 	&\mbox{(Ley de Faraday-Lenz)}		
	\label{seq:FL}\\
	\nabla \times\vb{B} &= \mu_0 \vb{J}_{\text{tot}} +\varepsilon_0\mu_0 \pdv{\vb{E}}{t}, &
	\mbox{(Ley de Ampère-Maxwell)} \label{seq:AM}
	\end{tcolorbox}\end{subequations}\noindent
%
donde $\rho_{\text{tot}}$ representa a la densidad de carga volumétrica y $\vb{J}_{\text{tot}}$ a la densidad de corriente volumétrica; $\epsilon_0$ a la permitividad eléctrica en el vacío y $\mu_0$ a la permeabilidad magnética en el vacío. 

En ausencia de fuentes externas (\( \rho_{\text{tot}} = 0 \), \( \vb{J}_{\text{tot}} = \vb{0} \)), los campos electromagnéticos pueden desacoplarse y satisfacer la ecuación de onda de Helmholtz al aplicar la transformada de Fourier temporal\footnote{Sea $F(t)$ una función real dependiente del tiempo, la transformada de Fourier $\mathcal{F}(\omega)$ de $F(t)$ se define como $\int_{-\infty}^{\infty}F(t)e^{i\omega t} \text{d}t$. Mientras que la transformada de Fourier inversa $\mathcal{F}^{-1}(t)$ es $1/{2\pi}\int_{-\infty}^{\infty}\mathcal{F}(\omega)e^{-i\omega t} \text{d}\omega$ \cite{arfkenMathematicalMethodsPhysicists2011a} } ~\cite{jacksonClassicalElectrodynamics2021a}. Una de las soluciones de esta ecuación son las ondas planas, que representan la propagación de una onda monocromática en una dirección definida

	\begin{subequations}%
	\eqhalf{\vb{E}(\vb{r},t) =\vb{E}_0 e^{i(\vb{k}\cdot\vb{r} -\omega t)},}\label{seq:E_plana}% 
	\eqhalf{\vb{B}(\vb{r}, t) =\vb{B}_0 e^{i(\vb{k}\cdot\vb{r} -\omega t)},}	\label{seq:B_plana}
	\label{eqs:ondasPlanas}\end{subequations}\vspace*{-1em}
		
\noindent donde $\vb{E}_0$ y $\vb{B}_0$ corresponden a las amplitudes de los campos, $\omega$ la frecuencia angular de la onda y $\vb{k}$ el vector de onda. Para que se satisfagan las Ecs. \eqref{eqs:ondasPlanas}, se tiene que cumplir la relación de dispersión dada por el número de onda $k=\sqrt{\mu\varepsilon}\;\omega$, donde $\varepsilon$ y $\mu$ corresponden a la permitividad eléctrica y la permeabilidad magnética del medio y son en general, funciones complejas dependientes de $\omega$ \cite{jacksonClassicalElectrodynamics2021a}. La relación de dispersión se puede reescribir en términos del índice de refracción del material dado por \cite{jacksonClassicalElectrodynamics2021a}
%
\begin{tcolorbox}[ams align]
	n(\omega) = \sqrt{\dfrac{\varepsilon\mu(\omega)}{\varepsilon_0\mu_0 }}.
	\label{eq:indice} 
\end{tcolorbox}
%	
\noindent con lo que se obtiene,
%
\begin{equation}
	k(\omega) =\dfrac{\omega n(\omega)}{c},
	\label{eq:numero_onda} 
\end{equation}

\noindent donde $c=1/\sqrt{\varepsilon_0\mu_0}$ es la velocidad de la luz en el vacío.

Al analizar la energía total almacenada en los campos electromagnéticos y el trabajo que estos realizan sobre una distribución de cargas y corrientes, se establece el teorema del trabajo y la energía. A partir de este teorema, se introduce el concepto de energía transportada por los campos por unidad de tiempo y por unidad de área, el cual está representado por el vector de Poynting, $\vb{S}$, dado por \cite{griffithsIntroductionElectrodynamics2023b}

\begin{tcolorbox}[ams align]
	\vb{S}=\frac{1}{\mu_0}(\vb{E}\times\vb{B}),
	\label{eq:vect_Poynting} 
\end{tcolorbox}


Las Ecs. \eqref{eqs:Maxwell} determinan los campos que surgen a partir de corrientes y cargas presentes en la materia. No obstante, dichas ecuaciones no explican el origen de las corrientes y cargas \cite{novotnyPrinciplesNanooptics2012a}. Por ello, es necesario complementar las Ecs. \eqref{eqs:Maxwell} con ecuaciones llamadas \textit{relaciones constitutivas}, que describen cómo responde la materia ante la acción de los campos. Para un medio lineal, homogéneo e isótropo, las relaciones constitutivas están dadas por \cite{novotnyPrinciplesNanooptics2012a}
%
\begin{subequations}\label{eqs:Constitutivas}
	\begin{tcolorbox}[ams align, breakable]
		\vb{D}(\vb{r},\omega) &= \epsilon(\omega)\; \vb{E}(\vb{r},\omega), \label{seq:D} \\
		\vb{B}(\vb{r},\omega) &= \mu(\omega) \vb{H}(\vb{r},\omega), \label{seq:B} \\
		\vb{J}(\vb{r},\omega) &= \sigma (\omega) \vb{E}(\vb{r},\omega), \label{seq:J}
	\end{tcolorbox}
\end{subequations}
%	
\noindent donde $\vb{D}$ corresponde al vector de desplazamiento eléctrico, $\vb{H}$ al campo H, $\vb{J}$ a la densidad volumétrica de corriente  y $\sigma$ corresponde a la conductividad eléctrica.

Dadas las relaciones anteriores, la parte temporal puede ser construida empleando la transformada de Fourier temporal. La transformada de Fourier temporal para $\vb{D}$ está dada por~\cite{jacksonClassicalElectrodynamics2021a}
%
\begin{equation}
	\vb{D}(\vb{r},\omega)=\int_{-\infty}^{\infty}\vb{D}(\vb{r},t')e^{i\omega t'}\text{d}t',\label{eq:TFI}
\end{equation}
%
mientras que la transformada de Fourier inversa está dada por
%
\begin{equation}
	\vb{D}(\vb{r},t)=\frac{1}{2\pi}\int_{-\infty}^{\infty}\vb{D}(\vb{r},\omega)e^{-i\omega t}\text{d}\omega.\label{eq:TF}
\end{equation}
%
Al sustituir la Ec. \eqref{seq:D} en la Ec. \eqref{eq:TF}, se obtiene
%
\begin{equation}
	\vb{D}(\vb{r},t)=\frac{1}{2\pi}\int_{-\infty}^{\infty}\varepsilon(\omega)\vb{E}(\vb{r},\omega)e^{-i\omega t}\text{d}\omega,\label{eq:TF_D_intermedia_1}
\end{equation}
%
y al sustituir la transformada de Fourier de $\vb{E}(\vb{r},t)$, se tiene que
%
\begin{equation}
	\vb{D}(\vb{r},t)=\frac{1}{2\pi}\int_{-\infty}^{\infty}\varepsilon(\omega)e^{-i\omega t}\text{d}\omega\int_{-\infty}^{\infty}e^{i\omega t'}\vb{E}(\vb{r},t')\text{d}t'.\label{eq:TF_D_intermedia_2}
\end{equation}
%
Considerando que los órdenes de integración se pueden intercambiar, dado que la integración se realiza en el mismo intervalo ($-\infty$, $\infty$), la Ec. \eqref{eq:TF_D_intermedia_2} se reescribe como
%
\begin{tcolorbox}[ams align]
	\vb{D}(\vb{r},t)=\frac{1}{2\pi}\int_{-\infty}^{\infty}\int_{-\infty}^{\infty}\varepsilon(\omega)\vb{E}(\vb{r},t') e^{i\omega (t-t')} \text{d}\omega \text{d}t'.\label{eq:TF_D_intermedia_3} 
\end{tcolorbox}
%
\noindent Al realizar el cambio de variable $\tau = t - t'$ y emplear la función constante \cite{arfkenMathematicalMethodsPhysicists2011a}
\begin{equation}
	\mathcal{F}[1]=\int_{-\infty}^{\infty}e^{-i\omega\tau}=\delta(\tau),
\end{equation}
donde $\delta(\tau)$ es la función delta de Dirac, que cumple con \cite{arfkenMathematicalMethodsPhysicists2011a}
\begin{equation}
	\int_{-\infty}^{\infty}f(\tau)\delta(\tau)\text{d}\tau=f(0),
\end{equation}
se reescribe al campo eléctrico como
\begin{equation}
	\vb{E}(\vb{r},t)=\int_{-\infty}^{\infty}\vb{E}(\vb{r},t-\tau)\delta(\tau)=\frac{1}{2\pi}\int_{-\infty}^{\infty}\int_{-\infty}^{\infty}\vb{E}(\vb{r},t-\tau)\e^{-i\omega\tau}\text{d}\omega\text{d}\tau.
\end{equation}
%
Al sumar $\vb{E}(\vb{r},t')-\vb{E}(\vb{r},t')$ y multiplicar por $\varepsilon_0/\varepsilon_0$ a la Ec. \eqref{eq:TF_D_intermedia_2}, simplificando se obtiene
%
\begin{tcolorbox}[ams align]
		\vb{D}(\vb{r},t)=\varepsilon_0\left[\vb{E}(\vb{r},t)+\int_{-\infty}^{\infty}G(\tau)\vb{E}(\vb{r},t-\tau)\text{d}\tau\right],\label{eq:TF_D_final} 
\end{tcolorbox}
%
\noindent donde $G(\tau)$ es la transformada de Fourier de la susceptibilidad eléctrica $\chi_e=\varepsilon(\omega)/\varepsilon_0-1$ \cite{jacksonClassicalElectrodynamics2021a}
%
\begin{tcolorbox}[ams align]
	G(\tau)=\frac{1}{2\pi}\int_{-\infty}^{\infty}\left[\frac{\varepsilon(\omega)}{\varepsilon_0}-1\right]e^{-i\omega\tau}\text{d}\omega,
	\label{eq:G_tdependent} 
\end{tcolorbox}
%
\noindent con $\varepsilon(\omega)/\varepsilon_0$ la permitividad eléctrica relativa. Al aplicar la transformada de Fourier a la Ec.~\eqref{eq:G_tdependent}, se obtiene una expresión para la permitividad eléctrica relativa
%
\begin{tcolorbox}[ams align]
	\frac{\varepsilon(\omega)}{\varepsilon_0}=1+\int_{-\infty}^{\infty}G(\tau) e^{-i\omega\tau}\text{d}\tau.
	\label{eq:epsrelativa} 
\end{tcolorbox}
%

\noindent Las Ecs. \eqref{eq:TF_D_final} y \eqref{eq:G_tdependent} muestran que el campo de desplazamiento eléctrico al tiempo $t$ depende del campo eléctrico en todos los demás tiempos $t'$. A esta característica, se le conoce como la no localidad temporal entre $\vb{D}$ y $\vb{E}$ \cite{jacksonClassicalElectrodynamics2021a}. Además, a partir de la Ec. \eqref{eq:G_tdependent}, se observa que si $\varepsilon(\omega)$ es independiente de $\omega$, $G(\tau)=\delta(\tau)$, por lo que se obtiene una respuesta instantánea, mientras que si sí depende de $\omega$, $G(\tau)$ es distinta de cero para todos los valores de $\tau$ distintos de cero.


De modo que las Ecs. \eqref{eq:TF_D_final} y \eqref{eq:epsrelativa} sean causales, es necesario imponer condiciones que garanticen que la Ec. \eqref{eq:G_tdependent} se anule para $\tau<0$, lo que se traduce en que al tiempo $t$, únicamente valores del campo eléctrico previos a ese tiempo determinan el vector de desplazamiento eléctrico \cite{jacksonClassicalElectrodynamics2021a}. De esta forma, las Ecs. \eqref{eq:epsrelativa} se reescriben como
%
\begin{align*}
	\vb{D}(\vb{r},t)&=\varepsilon_0\left[\vb{E}(\vb{r},t)+\int_{0}^{\infty}G(\tau)\vb{E}(\vb{r},t-\tau)\text{d}\tau\right],\\ \label{eq:TF_D_final}
	\frac{\varepsilon(\omega)}{\varepsilon_0}&=1+\int_{0}^{\infty}G(\tau) e^{-i\omega\tau}\text{d}\tau.
\end{align*}
	%


















