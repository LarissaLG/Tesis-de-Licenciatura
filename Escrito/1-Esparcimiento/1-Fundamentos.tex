% !TeX root = ../tesis.tex

\chapter{Esparcimiento de luz por partículas}
\label{chapter:theory}

\vspace*{7em}

En este capítulo


\section{Fundamentos}
\label{section:basics}
Las ecuaciones de Maxwell, junto con la fuerza de Lorentz\footnote{La fuerza de Lorentz es la fuerza que experimenta una partícula de carga \( q \) se desplaza con velocidad \( \vb{v} \) en presencia de un campo eléctrico \( \vb{E} \) y un campo magnético \( \vb{B} \). En unidades del Sistema Internacional está dada por $\vb{F}=q(\vb{E}+\vb{v}\times\vb{B})$ \cite{zangwillModernElectrodynamics2013}. }, describen a la electrodinámica clásica, que se centra en el origen y comportamiento de los campos electromagnéticos \cite{zangwillModernElectrodynamics2013}. En unidades del Sistema Internacional, dichas ecuaciones se expresan en forma diferencial como
\cite{griffithsIntroductionElectrodynamics2023b}
%
	\begin{subequations} \label{eqs:Maxwell}
	\begin{tcolorbox}[
	ams align, breakable]
	\nabla \cdot\vb{E} &= \frac{\rho_{\text{tot}}}{\varepsilon_0}, &\mbox{(Ley de Gauss eléctrica)}  
	\label{seq:GE} \\
	\nabla \cdot\vb{B} &= 0,						&\mbox{(Ley de Gauss magnética)}   
	\label{seq:GM} \\
	\nabla \times\vb{E} &= -\pdv{\vb{B}}{t}, 	&\mbox{(Ley de Faraday-Lenz)}		
	\label{seq:FL}\\
	\nabla \times\vb{B} &= \mu_0 \vb{J}_{\text{tot}} +\varepsilon_0\mu_0 \pdv{\vb{E}}{t}, &
	\mbox{(Ley de Ampère-Maxwell)} \label{seq:AM}
	\end{tcolorbox}\end{subequations}\noindent
%
donde se omiten las dependencias espaciales y temporales y $\vb{E}(\vb{r},t)$ es el campo eléctrico,  $\vb{B}(\vb{r},t)$ es el campo magnético; $\rho_{\text{tot}}$  es la densidad de carga volumétrica, $\vb{J}_{\text{tot}}$ la densidad de corriente volumétrica; $\epsilon_0$ es la permitividad eléctrica en el vacío y $\mu_0$ la permeabilidad magnética en el vacío. 

\noindent Si bien las Ecs. \eqref{eqs:Maxwell} determinan los campos generados a partir de cargas y corrientes, por sí solas no determinan la respuesta de las cargas contenidas en el interior de la materia ante la acción de dichos campos \cite{novotnyPrinciplesNanooptics2012a}. Para ello, es necesario complementar las Ecs. \eqref{eqs:Maxwell} con ecuaciones que proporcionen las propiedades electromagnéticas del medio dadas por \cite{griffithsIntroductionElectrodynamics2023b}
%
\begin{align}
	\vb{D}(\vb{r},t)&=\epsilon_0\vb{E}(\vb{r},t)+\vb{P}(\vb{r},t)\\
	\vb{H}(\vb{r},t)&=\frac{\vb{B}(\vb{r},t)}{\mu_0}-\vb{M}(\vb{r},t)
\end{align}
%
donde $\vb{D}$ es el vector de desplazamiento eléctrico, $\vb{P}$ la polarización; $\vb{H}$ es el campo H y $\vb{M}$ es la magnetización. La relación lineal más general entre $\vb{E}$ y $\vb{D}$ está dada por \cite{jacksonClassicalElectrodynamics2021a}
%
\begin{tcolorbox}[ams align]
	\vb{D}(\vb{r},t)=\int\int\stackrel{\leftrightarrow}{\varepsilon}(\vb{r}-\vb{r}',t-t')\vdot\vb{E}(\vb{r}',t')\dd{V}'\dd{t'}\
	\label{eq:d} 
\end{tcolorbox}
%	
\noindent donde $\stackrel{\leftrightarrow}{\varepsilon}$ es un tensor de rango dos y representa una función de respuesta\footnote{Las funciones de respuesta proporcionan información sobre el efecto de las interacciones electromagnéticas con el medio y sobre la estructura y propiedades del propio medio \cite{GeneralPropertiesElectromagnetic1989}.} del material en el espacio y el tiempo \cite{jacksonClassicalElectrodynamics2021a}.

\noindent La Ec. \eqref{eq:d} muestra que el campo de desplazamiento eléctrico al tiempo $t$ depende del campo eléctrico en todos los demás tiempos $t'$. Adicionalmente, el campo eléctrico de desplazamiento en un punto $\vb{r}$ depende de los valores del campo eléctrico en puntos vecinos $\vb{r}'$ \cite{jacksonClassicalElectrodynamics2021a}. A estas características, se les conoce como la no localidad temporal y espacial, respectivamente, entre $\vb{D}$ y $\vb{E}$ \cite{jacksonClassicalElectrodynamics2021a}.

En un medio isótropo y homogéneo, $\stackrel{\leftrightarrow}{\varepsilon}$ se reduce a una función escalar compleja $\varepsilon$ \cite{jacksonClassicalElectrodynamics2021a}; en este caso, tanto $\varepsilon$ como $\vb{E}$ son independientes de la posición y dependen únicamente de la diferencia temporal, es decir \cite{jacksonClassicalElectrodynamics2021a}
%
\begin{tcolorbox}[ams align]
	\vb{D}(t)=\int_{-\infty}^{\infty}\varepsilon(t-t')\vb{E}(t')\dd{t'},
	\label{eq:d1} 
\end{tcolorbox}
%	
\noindent de modo que la Ec. \eqref{eq:d1} es una convolución\footnote{La convolución de dos funciones $f(t), g(t)$ está dada por $f\ast g=\int_{-\infty}^{\infty}f(t-t')g(t')\dd{t'}$ \cite{arfkenMathematicalMethodsPhysicists2011a}.} en el tiempo. Al aplicar el teorema de la convolución\footnote{El teorema de la convolución es $\mathscr{F}[f\ast g]=\mathscr{F}[f]\mathscr{F}[g]$, donde $\mathscr{F}$ es la transformada de Fourier de una función. En esta tesis se considera la convención de la transformada de Fourier como \cite{arfkenMathematicalMethodsPhysicists2011a}
	\begin{align*}
		 f(\omega)&=\mathscr{F}[f(t)]=\int_{-\infty}^{\infty}f(t)e^{i\omega t}\dd{t},\\ f(t)&=\mathscr{F}^{-1}[f(\omega)]=\frac{1}{2\pi}\int_{-\infty}^{\infty}f(\omega)e^{i\omega t}\dd{\omega}.
	\end{align*}}
 y aplicar la transformada de Fourier, se obtiene \cite{bohrenAbsorptionScatteringLight2008}
%
\begin{tcolorbox}[ams align]
		\vb{D}(\omega)=\varepsilon(\omega)\vb{E}(\omega).
	\label{eq:d3} 
\end{tcolorbox}
%	
\noindent Así, en el espacio de frecuencias la respuesta del medio queda completamente caracterizada por una función dieléctrica dependiente de la frecuencia.

En ausencia de fuentes externas ($ \rho_{\text{tot}} = 0, \vb{J}_{\text{tot}} = \vb{0}$) y en un medio lineal, homogéneo e isótropo, las ecuaciones de Maxwell \eqref{eqs:Maxwell} pueden desacoplarse y, al aplicar la transformada de Fourier, se obtienen las ecuaciones de Helmholtz para $\vb{E}$ y para $\vb{B}$ \cite{jacksonClassicalElectrodynamics2021a}
%
\begin{equation}
	\laplacian{\vb{E}}+k^2\vb{E}=0\hspace{3cm}\text{y}\hspace{3cm} \laplacian{\vb{B}}+k^2\vb{B}=0.
	\label{eqs:helmholtz}
\end{equation}
%
Una posible solución son las ondas planas, en las que los campos son uniformes en cada plano perpendicular a la dirección de propagación \cite{griffithsIntroductionElectrodynamics2023b}
%
\begin{equation}
	\vb{E}(\vb{r},t) =\vb{E}_0 e^{i(\vb{k}\cdot\vb{r} -\omega t)}\hspace{2cm}\text{y}\hspace{2cm} \vb{B}(\vb{r}, t) =\vb{B}_0 e^{i(\vb{k}\cdot\vb{r} -\omega t)},	
	\label{eqs:ondas_planas}
\end{equation}
%
\noindent donde $\vb{E}_0$ y $\vb{B}_0$ corresponden a las amplitudes de los campos, $\omega$ es la frecuencia angular de la onda y $\vb{k}$ el vector de onda. Para que las Ecs. \eqref{eqs:ondas_planas} satisfagan las Ecs. \eqref{eqs:helmholtz}, se tiene que cumplir la relación de dispersión de una onda plana dada por el número de onda $k=\sqrt{\mu\varepsilon}\;\omega$, donde $\varepsilon$ y $\mu$ corresponden a la permitividad eléctrica y la permeabilidad magnética del medio y son en general, funciones complejas dependientes de $\omega$ \cite{jacksonClassicalElectrodynamics2021a}. La relación de dispersión se puede reescribir en términos del índice de refracción del material dado por \cite{jacksonClassicalElectrodynamics2021a}
%
\begin{tcolorbox}[ams align]
	n(\omega) = \sqrt{\dfrac{\varepsilon(\omega)\mu(\omega)}{\varepsilon_0\mu_0 }},
	\label{eq:indice} 
\end{tcolorbox}
%	
\noindent con lo que se obtiene
%
\begin{equation}
	k(\omega) =\dfrac{\omega n(\omega)}{c},
	\label{eq:numero_onda} 
\end{equation}

\noindent donde $c=1/\sqrt{\varepsilon_0\mu_0}$ es la velocidad de la luz en el vacío.

El flujo de energía que transporta una onda electromagnética en la dirección de propagación se describe mediante el vector de Poynting  $\vb{S}$ \cite{jacksonClassicalElectrodynamics2021a}
\begin{tcolorbox}[ams align]
	\vb{S}=\frac{1}{\mu_0}(\vb{E}\times\vb{B}).
	\label{eq:vect_Poynting} 
\end{tcolorbox}
\noindent El vector de Poynting tiene unidades de energía por unidad de área y por unidad de tiempo. Para una onda armónica, $\vb{E}$ y $\vb{B}$ oscilan en el tiempo, de modo que $\vb{S}$ también varía en el tiempo, por lo que se emplea el vector de Poynting promedio \cite{bohrenAbsorptionScatteringLight2008}
\begin{tcolorbox}[ams align]
	\langle\vb{S}\rangle_t = (1/2) \text{Re}[\vb{E} \times (\vb{B}/\mu)^{*}],
	\label{eq:vect_Poynting_prom_arm} 
\end{tcolorbox}
\noindent donde $*$ denota la operación complejo conjugado.














