\section{Modelado y propiedades ópticas}
\label{section:optical_properties}

The main characteristic of the erythrocytes’ shape is the presence of concavities on its sides. This makes an
exact light scattering calculation especially difficult. To overcome this problem some studies use simplified
shape models like flat cylinders with a rounded edge or oblate spheroids. Such approximations for example are used when light scattering is simulated by the standard T-matrix approach [6]. The problem of shape models’
influence has recently been investigated by Eremina et al. [5]

The erythrocyte has an advantage for modeling, as it has no internal
structure (like nucleon) and can be modeled as a homogeneous object with a certain refractive index. On the
other side light scattering simulation is difficult due to the fact that erythrocyte has a relatively large (with
respect to the exciting wavelength) size, which can vary from 4 to 9 mm in diameter and its main shape
characteristics: the natural shape of a strainless erythrocyte is a biconcave discoid. But the erythrocyte is
surrounded by a thin elastic membrane and can change its form from biconcave to toroidal or to spherical one
depending on outward conditions